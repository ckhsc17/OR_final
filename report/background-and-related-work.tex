%\subsubsection*{The Role of Group Deliberation in Digital Democracy}
Digital democracy platforms like vTaiwan seek to turn diverse public input into actionable consensus.
While online engagement collects wide-ranging views, small face-to-face deliberations help participants
exchange reasons, clarify misunderstandings, and negotiate trade-offs in real time—enhancing empathy,
reducing polarization, and strengthening trust.
These interpersonal processes are essential complements to algorithmic aggregation.

\subsubsection*{Empirical Findings on Deliberative Group Sizes}\label{subsec:group_sizes}

Studies show a trade-off between inclusiveness and interaction quality.
Small groups (3–5 members) promote balanced, interactive dialogue~\parencite{fay2000group},
while mid-sized groups (8–15) offer more diversity with manageable
coordination~\parencite{involve_citizensjury, fishkin2009deliberative}.
Very large groups often suffer from monologues and disengagement.
Odd-numbered group sizes reduce decision deadlocks and polarization~\parencite{menon2011oddgroups},
making sizes like 5, 7, or 9 both empirically and procedurally robust.

Assuming only about 10\% of the 1921 participants opt into in-person deliberation,
we expect roughly 15–30 groups of 3–13 participants each.

\subsubsection*{Challenges in Group Formation}

Naive clustering approaches (e.g., k-means) often ignore fairness and deliberative quality,
leading to echo chambers or imbalanced groups.
Such methods offer limited control over group sizes or stakeholder representation.

\subsubsection*{Optimization Approaches to Group Design}

Integer Programming (IP) and related techniques have been used for fair team formation and clustering
with explicit constraints~\parencite{anagnostopoulos2012groupformation, celis2018fair}.
Though computationally intensive, these approaches are tractable at vTaiwan’s scale and offer
fine-grained control over group diversity, size, and fairness.
