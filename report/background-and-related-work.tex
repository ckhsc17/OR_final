%\subsubsection*{The Role of Group Deliberation in Digital Democracy}
Digital Democracy platforms like vTaiwan aim to harness the collective intelligence of diverse participants.
Group deliberation serves as a vital complement to purely online input
by transforming fragmented digital opinions into collective judgments grounded in mutual understanding.
In digital democracy, small, structured face‐to‐face discussions allow participants to articulate reasoning,
confront misunderstandings, and negotiate trade‐offs in real time—processes that algorithms alone cannot replicate.
By fostering interpersonal trust and exposing individuals to diverse perspectives, in‐person deliberation mitigates
echo chambers and polarization, thereby enhancing the legitimacy and resilience of digitally informed policy decisions.

\subsubsection*{Empirical Findings on Deliberative Group Sizes}\label{subsec:group_sizes}

Empirical research consistently highlights a trade-off between inclusiveness and interaction quality.
Smaller groups facilitate mutual listening, balanced speaking opportunities, and greater engagement,
while larger groups offer broader representativeness at the cost of reduced dialogue quality
and increased coordination burdens.

Experimental studies demonstrate that groups with 3 to 5 participants tend to exhibit more interactive,
egalitarian discussions, avoiding patterns of domination or disengagement~\parencite{fay2000group}.
These \enquote{micro-groups} often yield higher-quality deliberation due to minimized social loafing
and production-blocking, phenomena well documented in small-group psychology.
Moderators frequently recommend configurations like \enquote{4$\pm$1} as optimal for promoting sustained interaction
and trust-building.

Conversely, empirical accounts of citizens’ juries, study circles, and deliberative assemblies reveal
that moderate group sizes—typically between 8 and 15—strike a practical balance between diversity of perspectives
and manageable interaction dynamics~\parencite{involve_citizensjury, participedia_citizensjury}.
Citizens’ juries often convene with 12 to 25 members,
while large-scale forums such as deliberative polls split hundreds of participants into smaller moderated tables
of approximately 8 to 12 participants~\parencite{fishkin2009deliberative}.
Importantly, deliberation tends to degrade in quality when group sizes exceed 10 to 12 participants,
leading to serial monologues and passive participation.

Additionally, psychological research emphasizes the importance of maintaining odd-numbered group sizes
to avoid any deadlocks in binary decisions~\parencite{menon2011oddgroups}.
Even-numbered groups are more prone to indecision or tense polarization, as tie outcomes lack a natural majority.
Consequently, odd group sizes—especially in the range of 5 to 13—are widely regarded as both empirically
and procedurally advantageous in structured deliberative formats.

When separating participants into groups of such small sizes, we assume that not all the 1921 participants
actually want to participate in the face-to-face deliberation.
If we assume that only about 10\,\% of the participants are willing to participate,
we could assume to end up with 15 to 30 groups.

\subsubsection*{Challenges in Group Formation}

Unstructured or similarity-based clustering (e.g., k-means, spectral clustering) may reinforce echo chambers,
lack diversity, or yield stakeholder imbalances.
These methods do not allow explicit constraints on group sizes or fairness,
making them ill-suited to participatory settings where equity and deliberative quality are paramount.

\subsubsection*{Optimization Approaches to Group Design}

Integer Programming (IP), Mixed Integer Linear Programming (MILP) and other sophisticated optimization techniques
have been used for team formation, peer review assignment,
and fair clustering~\parencite{anagnostopoulos2012groupformation, celis2018fair, charlin2013toronto}.
While computationally intensive, these models provide precision and flexibility.
For vTaiwan, where participant numbers are moderate and group sizes small, exact optimization is tractable and desirable.
