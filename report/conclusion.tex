By formulating the group formation challenge as an integer optimization problem based on Lagrangian decomposition,
we present a flexible and interpretable approach for assigning participants to deliberative groups.
Our model explicitly controls group size, diversity,
and engagement balance—capabilities missing in traditional clustering-based techniques.
The inclusion of a Linear Programming relaxation enables faster solution times for large-scale instances,
making the method practically applicable.
We expect the resulting groups to be more balanced and representative, improving discussion quality
and legitimacy in vTaiwan and similar participatory platforms.

\subsection*{Limitations and Future Work}
\begin{enumerate}
    \item \textbf{Developing a More Suitable Mathematical Model under OR Settings} \\
    In the current setting, not every participant votes on every comment. If we can collect more real-world data and increase the number of votes per comment, the model will become more robust. With richer data, we can better refine and adjust the hyperparameters to improve performance.

    \item \textbf{Collecting Real-World Feedback on the Group Assignment} \\
    We propose a new group assignment approach using OR techniques. To further enhance the model, it will be essential to gather real-world feedback from actual group assignment implementations. Such feedback can help identify practical challenges and guide improvements.

    \item \textbf{Implementing Dynamic Group Assignment} \\
    In real-world discussions, participants may join or leave at any time. Therefore, it would be valuable to design a dynamic system that continuously collects data and updates the group assignments in real time, ensuring adaptability and responsiveness.
\end{enumerate}