The \href{https://vtaiwan.tw}{vTaiwan} platform is a widely studied model of digital democracy,
combining online participation with structured, in-person deliberation to inform public policy in Taiwan.
Despite its success, vTaiwan faces persistent challenges such as \textit{group polarization},
\textit{unbalanced participation}, and \textit{inefficient consensus formation}.

For example, in the deliberation on
\href{https://blog.pol.is/uber-responds-to-vtaiwans-coherent-blended-volition-3e9b75102b9b}{Uber’s legality},
tech-savvy users and gig economy advocates dominated the discussion,
marginalizing traditional taxi drivers.
In the alcohol sales debate, participation skewed toward industry and libertarian voices,
with little input from public health stakeholders~\parencite{tiku2018taiwan}.
These cases illustrate how self-selection and opinion clustering can lead to echo chambers
and suppress diverse perspectives.

Currently, vTaiwan’s deliberation phase uses \href{https://pol.is/home}{Pol.is},
which clusters participants by opinion similarity.
While effective at mapping consensus, this approach can reinforce polarization
and fails to ensure stakeholder balance or engagement equity within groups.
To improve deliberative quality, we propose moving beyond organic clustering
toward structured group formation methods.

\subsubsection*{Context: Deliberation and Group Design}

Research in deliberative democracy shows that group composition
significantly affects discussion quality~\parencite{fay2000group}.
Small, diverse groups reduce risks of domination
and foster more robust consensus formation~\parencite{anagnostopoulos2012groupformation}.
Odd-numbered group sizes between 5 and 13 are especially effective,
balancing inclusiveness with coordination,
and avoiding decision deadlocks~\parencite{fishkin2009deliberative, menon2011oddgroups}.

Ensuring fairness requires deliberate attention to stakeholder affiliation, opinion diversity,
and participant engagement.
Unconstrained group formation can reproduce existing inequalities
or concentrate influence among dominant voices.

\subsubsection*{Project Overview}

This project introduces an optimization-based approach
to deliberative group formation for platforms like vTaiwan.
Instead of relying on emergent clustering,
we model group assignment as a constrained Integer Programming (IP) problem.
Each participant is represented by a feature vector derived from engagement metrics
(comment count, rating behavior, sentiment, stakeholder tags).
Our objective is to generate groups that:
\begin{enumerate}
  \item Maximize intra-group diversity;
  \item Prevent any stakeholder group from dominating too many groups;
  \item Balance engagement levels (and optionally sentiment) across groups.
\end{enumerate}
Group sizes are constrained to odd numbers from $\{3,5,7,9,11,13\}$.
We formulate a formal IP model to encode these goals and constraints,
and explore its linear relaxation to assess tractability and approximation quality.
