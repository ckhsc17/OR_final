The \href{https://vtaiwan.tw}{vTaiwan} system is an innovative digital democracy platform
designed to facilitate public policy deliberation in Taiwan.
It integrates online participation with structured, in-person discussions,
offering a hybrid model that has received international attention.
However, despite its successes, several persistent challenges have been identified:
\textit{group polarization}, \textit{unbalanced participation},
and \textit{inefficiencies in consensus formation}.

For instance, during the deliberation
on \href{https://blog.pol.is/uber-responds-to-vtaiwans-coherent-blended-volition-3e9b75102b9b}{Uber's legality},
technologically savvy users and gig economy supporters dominated the discussion,
creating an echo chamber that marginalized traditional taxi drivers.
Similarly, in the online debate over
\href{https://www.technologyreview.com/2018/08/21/240284/the-simple-but-ingenious-system-taiwan-uses-to-crowdsource-its-laws/}{alcohol sales},
participation was skewed towards industry representatives and libertarian advocates,
with minimal involvement from public health experts or concerned parents.
These cases highlight how self-selection and clustering by opinion can hinder inclusive deliberation
and suppress diverse perspectives.

Currently, the deliberative stage in vTaiwan is supported by \href{https://pol.is/home}{Pol.is},
which groups participants via natural clustering based on opinion similarity.
While effective for mapping consensus, this clustering method can reinforce polarization
and fails to guarantee stakeholder balance within discussion groups.
To ensure inclusive and high-quality deliberation, it is essential to move beyond purely organic grouping
and explore structured alternatives for group formation.

\subsubsection*{Context: Deliberation and Group Design}

Theoretical and empirical research in deliberative democracy emphasizes
the critical role of group composition in shaping the quality of
discourse~\parencite{fay2000group}.
Small, diverse groups are associated with richer argument exchange, lower risks of domination,
and more robust consensus
formation~\parencite{anagnostopoulos2012groupformation}.
Notably, group size matters: research shows that odd-sized groups between 5 and 13 participants
strike a balance between inclusiveness and manageability,
minimizing deadlocks while still representing multiple
perspectives~\parencite{fishkin2009deliberative}.

Ensuring fairness and representativeness in group composition requires attention to stakeholder affiliations,
opinion diversity, and participant engagement levels.
Without such design constraints, group formation may reproduce existing inequalities
or concentrate influence among dominant voices.
% TODO: for future work because very delicate:
% one should consider that there is a spectrum of two types of voices that need different balancing strategies:
% (1) those about facts or backed by hard evidence (like climate change),
% there the balance should be according to actual representation in the population/participants
% (2) those that are opinion-based (no evidence) or simply a matter of personal taste.
% There the balance should be such to also represent minorities equally to ensure a fair representation of all voices.

\subsubsection*{Project Overview}

This project proposes a formal optimization approach to deliberative group formation
for use in platforms such as vTaiwan.
Rather than relying on emergent clustering, we model the assignment
of participants to groups as a constrained Integer Programming (IP) problem.
Each participant is represented by a feature vector derived from their engagement
(e.g., comment count, rating frequency, sentiment, and inferred stakeholder set).
Our objective is to generate groupings that satisfy three principles:
\begin{enumerate}
  \item Maximize intra-group diversity;
  \item Ensure no stakeholder set dominates a significant majority of groups;
  \item Balance total engagement levels and optionally sentiment across groups.
\end{enumerate}
Group sizes are constrained to be odd numbers from $\{3,5,7,9,11,13\}$.
We formulate an IP model that encodes these goals and constraints,
and we also explore its linear relaxation to assess solution quality and approximation potential.
