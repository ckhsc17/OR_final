\subsubsection*{Summary and Justification of Design Choices}

This model balances ideological diversity and active participation in forming deliberative groups.
Constraints enforce odd-sized group sizes, regulate stakeholder diversity, and ensure fairness in participant engagement.
By integrating these factors into a single optimization framework, the model supports the formation of high-quality,
balanced groups well-suited for productive democratic deliberation.
Capping stakeholder dominance across groups safeguards against agenda capture
and supports procedural fairness in group composition.

\subsubsection*{Odd-Sized Groups}
Group sizes are restricted to odd values in the set $\{5,7,9,11,13\}$ to avoid deadlocks in binary decisions
and promote more inclusive, decisive deliberation~\parencite{menon2011oddgroups}.
See also Section~\ref{subsec:group_sizes} for empirical justification.

\subsubsection*{Promoting Diversity}
The objective function rewards heterogeneous group compositions based on precomputed dissimilarity scores $D(i,k)$
between participants. These scores can reflect ideological differences, issue preferences, or affective sentiment.

\subsubsection*{Fair Stakeholder Representation}
To avoid group capture by dominant interests, constraints limit the number of groups
in which any stakeholder identity forms a majority. This is controlled via a per-group dominance threshold $\alpha$
and a global cap $\gamma$ on the number of dominated groups.

\subsubsection*{Balanced Engagement}
Participants are scored using engagement metrics $E(i)$, reflecting prior activity (e.g., issues voted on, comments rated).
The model distributes engagement levels evenly across groups by minimizing deviations from a global target $\bar{E}$.
This helps prevent overconcentration of highly active users and encourages balanced group dynamics.


