Let:
\begin{itemize}
    \item $P = \{1, 2, \dots, n\}$: Set of participants.
    \item $G = \{1, 2, \dots, m\}$: Set of groups.
    \item $S = \{1, 2, \dots, r\}$: Set of stakeholder groups.
    \item $s(i) \in S$: Stakeholder group membership of participant $i$.
    \item $D(i,k)$: Diversity score between participants $i$ and $k$.\\(e.g., ideological distance.)
    \item $E(i)$: Engagement score of participant $i$.\\(e.g., number of distinct issues voted on.)
    \item $x_{ij} \in \{0,1\}$: Indicating whether participant $i$ is assigned to group $j$.
    \item $y_{jl} \in \{0,1\}$: Indicating whether group $j$ has size $l \in \{5,7,9,11,13\}$.
    \item $z_j \geq 0$: Auxiliary variable representing the absolute deviation of group $j$’s engagement from the target $\bar{E}$.
    \item $d_j \in \{0,1\}$: Binary variable indicating whether group $j$ is dominated by any stakeholder group.
\end{itemize}

Define group size as:
\[
s_j \coloneqq \sum_{i \in P} x_{ij}
\]

\subsection*{Objective Function}

We formulate a composite objective that balances intra-group diversity and engagement fairness:

\[
\max \left[
\lambda_1 \cdot \overbrace{
\sum_{j \in G} \sum_{\substack{i,k \in P \\ i < k}} D(i,k)\,x_{ij} x_{kj}
  }^{\text{(A) Intra-group Diversity}}
\hspace{0.2cm} - \hspace{0.2cm} \lambda_2 \cdot \hspace{-1.3cm} \overbrace{
\sum_{j \in G} z_j
}^{\text{(B) Engagement Imbalance}}
\right]
\]

\noindent
\textbf{(A) Intra-group Diversity:}\\
Rewards group compositions that contain participants with a range of differing perspectives.
Diversity scores \(D(i,k)\) are precomputed based on ideological dissimilarity or voting disagreement.
\\
\textbf{(B) Engagement Imbalance:}\\
Penalizes uneven distribution of participant engagement across groups by minimizing the total absolute deviation
from a target engagement level \(\bar{E}\).
\\
\textbf{Hyperparameter Tuning:}\\
To interpret and calibrate \(\lambda_1\) and \(\lambda_2\), we analyze the individual objective components (A)
and (B) independently.
Each term is normalized (e.g., via max-min or standard deviation scaling)
to ensure their relative importance is interpretable and tunable.

\subsection*{Constraints}
\\
\textbf{(1) Unique Assignment}\\
Each participant must be assigned to exactly one group:
\[
\sum_{j \in G} x_{ij} = 1 \quad \forall i \in P
\]
\\
\textbf{(2) Discrete Group Sizes}\\
Each group must have a size from the set \(\{5,7,9,11,13\}\).
We enforce this using:
\begin{gather*}
    s_j = \sum_{i \in P} x_{ij} \quad \forall j \in G\\
    \sum_{l \in \{5,7,9,11,13\}} y_{jl} = 1 \quad \forall j \in G\\
    s_j = \sum_{l \in \{5,7,9,11,13\}} l \cdot y_{jl} \quad \forall j \in G\\
\end{gather*}
\textbf{(3) Diversity Threshold}\\
To avoid excessive internal conflict, each group’s diversity must not exceed a maximum threshold:
\[
\sum_{i < k} D(i,k)\,x_{ij} x_{kj} \leq \delta \quad \forall j \in G
\]
\textbf{(4) Minimum Engagement per Group}\\
Each group must contain a minimum total engagement score:
\[
\sum_{i \in P} E(i)\,x_{ij} \geq \eta \quad \forall j \in G
\]
\textbf{(5) Engagement Deviation Linearization}\\
We define the absolute deviation of each group’s engagement from the target \(\bar{E}\) using auxiliary variables:
\begin{gather*}
    \sum_{i \in P} E(i)\,x_{ij} - \bar{E} \leq z_j \quad \forall j \in G\\
    \bar{E} - \sum_{i \in P} E(i)\,x_{ij} \leq z_j \quad \forall j \in G\\
\end{gather*}
\textbf{(6) Stakeholder Dominance Control}\\
To prevent any stakeholder group from dominating too many groups, we restrict the number of groups
in which more than a threshold fraction \(\alpha\) of members come from the same stakeholder group.

Let \(I_\sigma = \{i \in P \mid s(i) = \sigma\}\) denote the set of participants from stakeholder group \(\sigma \in S\).
Then:
\begin{gather*}
    \sum_{i \in I_\sigma} x_{ij} \leq \alpha \cdot s_j + M \cdot d_j \quad \forall j \in G,\ \forall \sigma \in* S\\
    \sum_{j \in G} d_j \leq \gamma\\
\end{gather*}
\vspace{-0.5cm}
Where:
\begin{itemize}
    \item \(\alpha \in (0.5, 1)\): Dominance threshold (e.g., \(\alpha = 0.5\) for majority control)
    \item \(d_j \in \{0,1\}\): Indicator whether group \(j\) is dominated
    \item \(M\): Large constant (e.g., \(M = 13\)) to linearize the conditional logic
    \item \(\gamma\): Maximum of allowed dominated groups (e.g., \(\gamma = \lceil 0.05 \cdot |G| \rceil\))
\end{itemize}

\subsection*{Constraint Threshold Calibration}
\\
\textbf{Diversity Threshold \(\delta\):}\\
Empirically selected from the distribution of intra-group diversity scores in observed or simulated groupings.
Typical values include the 75th or 90th percentile of feasible groupings.
\\
\textbf{Engagement Threshold \(\eta\):}\\
Derived from desired minimum engagement per group.
For example, if three highly active participants per group are desired,
set \(\eta = 3 \cdot \bar{E}_{\text{individual}}\).

\subsection*{Optional LP Relaxation}

To improve scalability, we relax the binary and discrete constraints as follows:
\begin{itemize}
    \item Assignment variables: \quad $x_{ij} \in \{0,1\}$ is relaxed to $x_{ij} \in [0,1]$.
    \item Group sizes: \quad $s_j \in \{5,7,9,11,13\}$ is relaxed to $5 \leq s_j \leq 13$.
    \item Dominance indicators: \quad $d_j \in \{0,1\}$ can be relaxed to $d_j \in [0,1]$ in LP versions with appropriate rounding afterward.
\end{itemize}
We may apply rounding and post-processing to recover feasible integral solutions from the relaxed model.
