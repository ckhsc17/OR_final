Let:
\begin{itemize}
    \item $P = \{1, 2, \dots, n\}$: Set of participants.
    \item $G = \{1, 2, \dots, m\}$: Set of groups.
    \item $x_{ij} \in \{0,1\}$: Indicating whether participants $i$ is assigned to group $j$.
    \item $y_{jl} \in \{0,1\}$: Indicating whether group $j$ has size $l \in \{5,7,9,11,13,15\}$.
    \item $D(i,k)$: Diversity score between participants $i$ and $k$ (e.g., ideological distance.)
    \item $E(i)$: Engagement score of participants $i$ (e.g., number of comments voted on.)
    \item $z_j \geq 0$: Auxiliary variable for group $j$’s absolute deviation from the target engagement $\bar{E}$.
\end{itemize}

\noindent
Additionally we define group size as $s_j \coloneqq \sum_{i \in P} x_{ij}$.

\subsection*{Objective Function}

We formulate a composite objective that balances intra-group diversity and engagement fairness:

\[
\max \left[
\lambda_1 \cdot \overbrace{
\sum_{j \in G} \sum_{\substack{i \in P, k \in P \\ i < k}} D(i,k)\,x_{ij} x_{kj}
  }^{\text{(A) Intra-group Diversity}}
\hspace{0.2cm} - \hspace{0.2cm} \lambda_2 \cdot \hspace{-1.3cm} \overbrace{
\sum_{j \in G} z_j
}^{\text{(B) Engagement Imbalance}}
\right]
\]

\noindent
\textbf{(A) Intra-group Diversity:}\\[3pt]
Rewards group compositions that contain participants with a range of differing perspectives.
Diversity scores \(D(i,k)\) are precomputed based on ideological dissimilarity or voting disagreement.
\\
\textbf{(B) Engagement Imbalance:}\\[3pt]
Penalizes uneven distribution of participant engagement across groups by minimizing the total absolute deviation
from a target engagement level \(\bar{E}\).
\\
\textbf{Hyperparameter Tuning:}\\[3pt]
To interpret and calibrate \(\lambda_1\) and \(\lambda_2\), we analyze the individual objective components (A)
and (B) independently.
Each term is normalized (e.g., via max-min or standard deviation scaling)
to ensure their relative importance is interpretable and tunable.

\subsection*{Constraints}
\textbf{(1) Unique Assignment}\\[3pt]
Each participant must be assigned to exactly one group:
\[
\sum_{j \in G} x_{ij} = 1 \quad \forall i \in P
\]
\\
\textbf{(2) Discrete Group Sizes}\\[3pt]
Each group must have a size from the set \(\{5,7,9,11,13,15\}\).
We enforce this using:
\begin{gather*}
    s_j = \sum_{i \in P} x_{ij} \quad \forall j \in G\\
    \sum_{l \in \{5,7,9,11,13,15\}} y_{jl} = 1 \quad \forall j \in G\\
    s_j = \sum_{l \in \{5,7,9,11,13,15\}} l \cdot y_{jl} \quad \forall j \in G
\end{gather*}
\textbf{(3) Diversity Threshold}\\[3pt]
To avoid excessive internal conflict, each group's diversity must not exceed a maximum threshold:
\[
\sum_{i < k} D(i,k)\,x_{ij} x_{kj} \leq \delta \quad \forall j \in G
\]
\textbf{(4) Minimum Engagement per Group}\\[3pt]
Each group must contain a minimum total engagement score:
\[
\sum_{i \in P} E(i)\,x_{ij} \geq \eta \quad \forall j \in G
\]
\textbf{(5) Engagement Deviation Linearization}\\[3pt]
We define the absolute deviation of each group's engagement from the target \(\bar{E}\) using auxiliary variables:
\begin{gather*}
    \sum_{i \in P} E(i)\,x_{ij} - \bar{E} \leq z_j \quad \forall j \in G\\
    \bar{E} - \sum_{i \in P} E(i)\,x_{ij} \leq z_j \quad \forall j \in G
\end{gather*}

\textbf{(6) Stakeholder Dominance Control}\\[3pt]
Participants are clustered into a fixed number of \emph{stakeholder clusters}
based on similarities in voting behavior and preferences.
These represent shared interests (such as \textit{taxi drivers}, \textit{Uber drivers}, or \textit{customers}).

To prevent overrepresentation, we limit how often
a stakeholder cluster may form a relative majority within deliberation groups.
Without this constraint, the algorithm could produce too many groups dominated by a single stakeholder,
allowing them to disproportionately influence the outcome.
This mirrors \emph{gerrymandering} in political science~\parencite{gerrymandering2019},
where electoral boundaries are drawn to grant one party majority control over most districts.

\begin{itemize}
    \item $S = \{1, \dots, r\}$: Stakeholder clusters
    \item $s(i) \in S$: Stakeholder membership of participant $i \in P$
    \item $d_{j\sigma} \in \{0,1\}$: 1 if stakeholder $\sigma$ dominates group $j$
    \item $d_j \in \{0,1\}$: 1 if any stakeholder dominates group $j$
    \item $\alpha \in [0.5, 1)$: Maximum allowed proportion of any single stakeholder in a group
    \item $M \gg 0$: Large constant for constraint relaxation
\end{itemize}

Let \(I_\sigma = \{i \in P \mid s(i) = \sigma\}\) denote the set of participants from stakeholder group \(\sigma \in S\).
Then:
\begin{gather*}
    \sum_{i \in I_\sigma} x_{ij} \leq \alpha \cdot s_j + M \cdot d_j \quad \forall j \in G,\ \forall \sigma \in S,
    \qquad    \sum_{j \in G} d_j \leq \gamma
\end{gather*}
\noindent

\subsection*{Constraint Threshold Calibration}
\textbf{Diversity Threshold \(\delta\):}\\[3pt]
Empirically selected from the distribution of intra-group diversity scores in observed or simulated groupings.
Typical values include the 75th or 90th percentile of feasible groupings.
\\
\noindent
\textbf{Engagement Threshold \(\eta\):}\\[3pt]
Derived from desired minimum engagement per group.
For example, if three highly active participants per group are desired,
set \(\eta = 3 \cdot \bar{E}_{\text{individual}}\).

\subsection*{Optional LP Relaxation}

To improve scalability, we relax the binary and discrete constraints as follows:
\begin{itemize}
    \item Assignment variables: \quad $x_{ij} \in \{0,1\}$ is relaxed to $x_{ij} \in [0,1]$.
    \item Group sizes: \quad $s_j \in \{5,7,9,11,13,15\}$ is relaxed to $5 \leq s_j \leq 15$.
    \item Dominance indicators: \quad $d_j \in \{0,1\}$ can be relaxed to $d_j \in [0,1]$ in LP versions with appropriate rounding afterward.
\end{itemize}
We may apply rounding and post-processing to recover feasible integral solutions from the relaxed model.
